\documentclass{article}
\usepackage[utf8]{inputenc}
\usepackage{spaced}
\usepackage[default]{cantarell}
\usepackage[T1]{fontenc}
\title{Science Notes}
\author{Layton Wen}
\date{Period 1}

\begin{document}

\maketitle
\toc
\clearpage
\section{Erosion by Glaciers}
Glaciers cause erosion in two main ways:
\begin{definition}[Plucking]
\textbf{Plucking} is the process of rocks and sediments being picked up and frozen to the bottom of a glacier.
\end{definition}
\begin{definition}[Abrasion]
\textbf{Abrasion} is the process where glaciers use sediments and rocks stuck to the bottom of the glacier by plucking to scrape up rock. Sometimes, glaciers leave grooves which show the direction the glacier was moving, these are called \textbf{glacial striations}.
\end{definition}
\subsection{Valley Glaciers}
Valley glaciers are glaciers which travel through valleys(duh).\\[5pt] In particular, when a glacier travels through a \textbf{V-shaped valley}, it scrapes the edges of the valley, effectively changing the V-shaped valley into a \textbf{U-shaped valley}. Two instances where this has occured is shown below.
\begin{center}
    \includegraphics[width=12cm]{ushaped.png}
\end{center}
\textbf{Hanging valleys} can be carved out from other valleys, as seen in the below figure.
\begin{center}
    \includegraphics[width=8cm]{hanging valley.png}
\end{center}
\textbf{Cirques} are hollow "mini-valleys" carved out by glaciers atop a mountain. 
\begin{center}
    \includegraphics[width=14cm]{f-d_ae1dd6d3ca4e46dd173f7bfa1723baea985b4c7592d1e81713d466c1+IMAGE_THUMB_POSTCARD_TINY+IMAGE_THUMB_POSTCARD_TINY.png}
\end{center}
When cirques form on opposite sides of a mountain, they leave a jagged ridge called an \textbf{arête}.\\[5pt]
\textbf{Horns} are formed atop mountains where all sides of the mountain have been eroded.
\begin{center}
    \includegraphics[width=8cm]{horn.png}
\end{center}
\subsection{Source(s)}
\href{https://flexbooks.ck12.org/cbook/ck-12-middle-school-earth-science-flexbook-2.0/section/14.10/primary/lesson/erosion-by-glaciers-ms-es}{CK-12}
\end{document}
