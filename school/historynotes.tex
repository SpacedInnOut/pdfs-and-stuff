\documentclass{article}
\usepackage[utf8]{inputenc}
\usepackage{spaced}
\usepackage[default]{cantarell}
\usepackage[T1]{fontenc}
\title{History Notes}
\author{Layton Wen}
\date{Period 6}

\begin{document}

\maketitle
\toc
\clearpage
\setcounter{section}{5}
\section{West African Empires}
\subsection{Empire of Ghana}
\subsection{Empire of Mali}
\subsection{Empire of Songhai}
\subsection{Historical and Artistic Traditions}
Major early African civilizations did not have a written language, instead they have an oral history. Griots were entrusted to remember African history and pass it down.
\begin{definition}[Oral History]
Oral history is a spoken record of past events, rather than a written record.
\end{definition}
\begin{definition}[Griots]
A griot is a west African storyteller.
\end{definition}
Griots told entertaining and informative stories, sometimes even acting them out in a play. However, sometimes the griots confused dates and names. \\[5pt]
Griots also told proverbs to teach lessons to the people.
\subsubsection{West African Proverbs}
\begin{itemize}
    \item "By crawling a child learns to stand" - Hema proverb
    \item "Wisdom is like fire. People take it from others." - African proverb(not specified exactly which civilization)
    \item "An army of sheep led by a lion can defeat an army of lions led by a sheep." - Ghanian Proverb
    \item "If you are filled with pride, then you will have no room for wisdom." - African proverb(not specified exactly which civilization)
    \item "Do not forget what it is to be a sailor because of being a captain yourself." - Tanzanian Proverb
    \item "He who learns, teaches." - Ethiopian Proverb
    \item "If you want to go quickly, go alone. If you want to go far, go together." - African Proverb(not specified exactly which civilization)
    \item "Sticks in a bundle are unbreakable." - Bondei Proverb
    \item "Show me your friend and I will show you your character." - African Proverb(not specified exactly which civilization)
    \item "You should not hoard your money and die of hunger." - Ghanian Proverb
    \item "The most beautiful fig may contain a worm." - Zulu Proverb
    \item "Always being in a hurry does not prevent death, neither does going slowly prevent living." - Ibo Proverb 
\end{itemize}
Griots sometimes told stories through epics. These stories were long poems about kingdoems and heroes. Two famous epic collections are the \textit{Dausi} and the \textit{Sundiata}.\\[5pt]
\subsubsection{Myths and Epics}
The \textit{Dausi} talks about Ghana's history, however many of the events that happened in the \textit{Dausi} are myths, though some are not. 
\begin{example}[Dausi]An example of one of these myths is a seven headed snake god, Bida. Bida said that Ghana would prosper as long as the people of Ghana sacrificed a young woman to him every year, but eventually a warrior killed Bida. However, ever since, Ghana had been cursed by the god, which caused Ghana to fall. \end{example}
The \textit{Sundiata} is also like the \textit{Dausi} except it talks about Mali. As the title suggests, the story is about Sundiata, Mali's first ruler. 
\begin{example}[Sundiata]According to the \textit{Sundiata}, a conqueror captured Mali and killed Sundiata's father and 11 brothers but spared Sundiata. Sundiata eventually overthrough this conqueror and became king of Mali.\end{example}
\subsubsection{Visitor records}
Although the people of West Africa didn't have their own written records of their history, visitors did. One of the first people to write about West Africa was an Arab scholar named \textbf{al-Masuid}. He talked about West Africa's geography, customs, history, and scientific achievements. Another prominent writer, \textbf{Abu Ubayd al-Bakri}, wrote about West Africa around 100 years later. However, the most famous person who wrote about West Africa was \textbf{Ibn Battutah}. He explored West Africa from 1353 to 1354, and he talks about West AFrica's political and cultural customs. Lastly, \textbf{Leo Africanus} traveled through North and West Africa on missions for the government of Spain. He wrote down what he had seen there, and shared it with the people of Rome.
\subsubsection{Sculpture}
Sculpture was the most well known of all West African visual art forms. Africans used wood, brass, clay, ivory, stone, and other materials to create beautiful statues.
\subsubsection{Masks}
Africans used wood to make masks that looked like animal faces. Some commonly used animal faces are hyenas, lions, monkeys, and antelopes.
\subsubsection{Clothing}
Many African Societies were well known for their cloth. Kente is the most famous cloth. Kente is made of brightly colored fabric, tightly woven together.
\subsubsection{Music and Dance}
Singing and dancing were for entertainment, but they were also for honoring African history and used in many celebrations. 
\end{document}
